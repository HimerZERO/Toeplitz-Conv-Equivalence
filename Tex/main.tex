%% ============================================
%% ================ Preambule =================
%% ============================================
\documentclass[]{scrartcl}
\usepackage[margin = 0.5in]{geometry}

\usepackage[pdftex,unicode,
colorlinks=true,
linkcolor = blue]{hyperref}	% нумерование страниц, ссылки!!!!ИМЕННО В ТАКОМ ПОРЯДКЕ СО СЛЕДУЮЩИМ ПАКЕТОМ
%\usepackage[warn]{mathtext}				% Поддержка русского текста в формулах
\usepackage[T1, T2A]{fontenc}			% Пакет выбора кодировки и шрифтов
\usepackage[utf8]{inputenc} 			% любая желаемая кодировка
\usepackage[english]{babel}		% поддержка русского языка
\usepackage{wrapfig}					% Плавающие картинки
\usepackage{amssymb, amsmath}			% стилевой пакет для формул
\usepackage{algorithm}
\usepackage{algorithmic}


\ifpdf
\usepackage{cmap} 				% чтобы работал поиск по PDF
\usepackage[pdftex]{graphicx}
%\usepackage{pgfplotstable}		% Для вставки таблиц.
\pdfcompresslevel=9 			% сжимать PDF
\else
\usepackage{graphicx}
\fi

\usepackage{subcaption}
%% ============================================
%% ================ Info =================
%% ============================================
\title{Свёрточные операции и Тёплицевы матрицы: анализ эквивалентности в нейронных сетях}
\author{\begin{tabular}{c}
	  	 Чернышов Игнат \\
		 \texttt{chernyshov.im@phystech.edu}
		\end{tabular}}
\date{\today}

\begin{document}

\maketitle

\begin{abstract}

Данный проект исследует связь между операциями свёртки в нейронных сетях и умножением на Тёплицевы матрицы. Основная часть проекта заключается в исследовании уже имеющихся результатов статьи <<LU decomposition and Toeplitz decomposition of a neural network>> и их локальной проверке. 

\end{abstract}

\section{Идея}

Ключевая идея заключается в том, что свёрточные слои нейронных сетей можно представить как умножение на специальные Тёплицевы матрицы. Это представление позволяет:

\begin{itemize}
\item Упростить теоретический анализ свёрточных операций
\item Применить методы линейной алгебры для оптимизации вычислений
\item По-новому взглянуть на свойства нейронных сетей
\end{itemize}


\bibliographystyle{unsrt}
\bibliography{biblio}

\end{document}
